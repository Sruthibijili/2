%iffalse
\let\negmedspace\undefined
\let\negthickspace\undefined
\documentclass[journal,12pt,twocolumn]{IEEEtran}
\usepackage{cite}
\usepackage{amsmath,amssymb,amsfonts,amsthm}
\usepackage{algorithmic}
\usepackage{graphicx}
\usepackage{textcomp}
\usepackage{xcolor}
\usepackage{txfonts}
\usepackage{listings}
\usepackage{enumitem}
\usepackage{mathtools}
\usepackage{gensymb}
\usepackage{comment}
\usepackage[breaklinks=true]{hyperref}
\usepackage{tkz-euclide} 
\usepackage{listings}
\usepackage{gvv}                                        
%\def\inputGnumericTable{}                                 
\usepackage[latin1]{inputenc}                    
\usepackage{color}                                            
\usepackage{array}                                            
\usepackage{longtable}                                       
\usepackage{calc}                                             
\usepackage{multirow}                                         
\usepackage{hhline}                                           
\usepackage{ifthen}                                           
\usepackage{lscape}
\usepackage{tabularx}
\usepackage{array}
\usepackage{float}
\newtheorem{theorem}{Theorem}[section]
\newtheorem{problem}{Problem}
\newtheorem{proposition}{Proposition}[section]
\newtheorem{lemma}{Lemma}[section]
\newtheorem{corollary}[theorem]{Corollary}
\newtheorem{example}{Example}[section]
\newtheorem{definition}[problem]{Definition}
\newcommand{\BEQA}{\begin{eqnarray}}
\newcommand{\EEQA}{\end{eqnarray}}
\newcommand{\define}{\stackrel{\triangle}{=}}
\theoremstyle{remark}
\newtheorem{rem}{Remark}
% Marks the beginning of the document
\begin{document}
\bibliographystyle{IEEEtran}
\vspace{3cm}
\title{Sequences and Series}
\author{EE24BTECH11060-Sruthi Bijili}
\maketitle
\newpage
\bigskip
\renewcommand{\thefigure}{\theenumi}
\renewcommand{\thetable}{\theenumi}
JEE ADVANCED/IIT-JEE\\

\begin{enumerate} [start=5]
    \item Sum of the $n$ terms of the series $\frac{1}{2}$$+$$\frac{3}{4}$$+$$\frac{7}{8}$$+$$\frac{15}{16}$$+$\dots is equal to
    \hfill(1988-2 Marks)
    \begin{enumerate}
        \item $2^n-n-1$
        \item $1-2^-2$
        \item $n+2^-2-1$
        \item $2^n+1$
         \end{enumerate}
    \item  The number $ \log_27$ is
    \hfill(1990- 2 Marks)
    \begin{enumerate}
        \item an integer
        \item a rational number
        \item an irrational number
        \item a prime number
    \end{enumerate}
    \item  If $\ln\brak{a+c}$,$\ln\brak{a+c}$,$\ln\brak{2b+c}$ are in A.P,then
    \hfill(1994)
    \begin{enumerate}
        \item a,b,c are in A,P
        \item $a^2$,$b^2$,$c^2$ are in A.P
        \item a,b,c are in G.P
        \item a,b,c are in H.P
    \end{enumerate}
    \item Let $a_{1}$,$a_{2}$,$a_{3}$\dots$a_{10}$ be in A.P,and $h_{1}$,$h_{2}$,$h_{3}$\dots$h_{10}$ be in H.P.If $a_{1}$$=$$h_{1}$$=$$2$ and $a_{10}$$=$$h_{10}$$=$$3$,then $a_{7}$$h_{7}$ is
    \hfill(1999 - 2 Marks)
    \begin{enumerate}
        \item $2$
        \item $3$
        \item $5$
        \item $6$
    \end{enumerate}
    \item The harmonic mean of the roots of the equation $\brak{5+\sqrt{2}}x^2$$-$$\brak{4+\sqrt{5}}x$$+$$8$$+$$2\sqrt{5}$$=$$0$ is 
    \hfill(1999 -2 Marks)
    \begin{enumerate}
        \item $2$
        \item $4$
        \item $6$
        \item $6$
    \end{enumerate}
    \item Consider an infinite geometric series with first term $a$ and common ratio $r$.If its sum is$4$ and the second term is $\frac{3}{4}$
    \hfill(2000S)
    \begin{enumerate}
        \item $a=\frac{4}{7}$,$r=\frac{3}{7}$
        \item $a=2$,$r=\frac{3}{8}$
        \item $a=\frac{3}{2}$,$r=\frac{1}{2}$
        \item $a=3$,$r=\frac{1}{4}$
    \end{enumerate}
    \item Let $\alpha$,$\beta$ be the roots of $x^2-x+p$$=$$0$ and $\gamma $,$\delta$ be the roots of $x^2-4x+q$$=$$0$.If$ \alpha$,$\beta$,$\gamma$,$\delta$ are in G.P,then the integral values of $p$ and $q$ respectively are
    \hfill(2001S)
    \begin{enumerate}
        \item $-2,-32$
        \item $-2,3$
        \item $-6,3$
        \item $6,-32$
    \end{enumerate}
    \item Let the ositive numbers $a$,$b$,$c$,$d$ be in A.P.Then $abc$, $abd$, $acd$, $bcd$ are
    \hfill(2001S)
    \begin{enumerate}
        \item NOT  in A.P/G.P/H.P
        \item in A.P
        \item in G.P
        \item in H.P
    \end{enumerate}
    \item If the sum  of the first $2n$ terms of the A.P $2,5,8,\dots$ is equal to the sum of the first $n$terms of the A.P,$57,59,61\dots$then $n$ equals
    \hfill(2001S)
    \begin{enumerate}
        \item $10$
        \item $12$
        \item $11$
        \item $13$
    \end{enumerate}
    \item  Suppose $a$,$b$,$c$ are in A.P and $a^2$,$b^2$,$c^2$ are in G.P,if $a$\textless$b$\textless$c$ and $a$$+$$b$$+$$c$$=$$\frac{3}{2}$, then the value of $a$ is 
    \hfill(2002S)
    \begin{enumerate}
        \item $\frac{1}{2\sqrt{2}}$
        \item $\frac{1}{2\sqrt{3}}$
        \item $\frac{1}{2}-\frac{1}{\sqrt{3}}$
        \item $\frac{1}{2}-\frac{2}{\sqrt{2}}$
    \end{enumerate}
    \item An infinite G.P has first term $'x'$ and sum $'5'$,then $x$ belongs to 
    \hfill(2004S)
    \begin{enumerate}
        \item x\textless$-10$
        \item 10\textless x \textless$0$
        \item $0$\textless x \textless$10$
        \item x\textgreater$0$
    \end{enumerate}
    \item In the quadratic equation $ax^2+bx+c$$=$$0$,$\Delta$$=$$b^2-4ac$ and $\alpha$$+$$\beta$,$\alpha^2$$+$$\beta^2$,$\alpha^3$$+$$\beta^3$, are in G.P where $\alpha$,$\beta$ are the root of $ax^2+bx+c$$=$$0$,then
    \hfill(2005S)
    \begin{enumerate}
        \item $\Delta$ $\neq$ $0$
        \item $b\Delta$ $\neq$ $0$
        \item $c\Delta$ $\neq$ $0$
        \item $\Delta$ $=$ $0$
    \end{enumerate}
    \item In the sum of first $n$ terms of an A.P is $cn^2$, then the sum of squares of these $n$  terms is 
    \hfill(2009)
    \begin{enumerate}
        \item $\frac{n\brak{4n^2-1}c^2}{6}$
        \item $\frac{n\brak{4n^2+1}c^2}{3}$
        \item $\frac{n\brak{4n^2-1}c^2}{3}$
        \item $\frac{n\brak{4n^2+1}c^2}{6}$
    \end{enumerate}
    \item Let $a_{1}$$,a_{2}$,$a_{3}$ \dots be in harmonic progression with $a_{1}$$=$$5$ and $a_{20}$$=$$25$.The least positive integer n for which $a_{n}$\textless$0$ is
    \hfill(2012)
    \begin{enumerate}
        \item $22$
        \item $23$
        \item $24$
        \item $25$
    \end{enumerate}
    \item Let $b_{i}$\textgreater$1$ for i$=$$1,2$,\dots$101$.Suppose $\log_eb_{1}$,$\log_eb_{2}$,\dots$\log_eb_{101}$ are in Arithmetic Progression \brak{A.P} with the common difference $\log_e2$.Suppose $a_{1}$,$a_{2}$,\dots$a_{101}$ are in A.P.such that $a_{1}$,$=$$b_{1}$ and $a_{51}$$=$$b_{51}$.If t$=$$b_{1}$$+$$b_{2}$$+$\dots$+$$ and b_{51}$ s$=$$a_{1}$$+$$a_{2}$$+$\dots$a_{53}$, then 
    \hfill(JEE ADV.2016)
    \begin{enumerate}
        \item s\textgreater t and $a_{101}$\textgreater$b_{101}$
        \item s\textgreater t and $a_{101}$\textless$b_{101}$
        \item s\textless t and $a_{101}$\textgreater$b_{101}$
        \item s\textless and $a_{101}$\textless$b_{101}$
    \end{enumerate}
    
\end{enumerate}




    


\end{document}
p
